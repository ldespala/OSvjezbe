\section{Shell skripte}

%\subsection*{Što su shell skripte i čemu služe?}
\textbf{Skripta} je datoteka koja sadrži naredbe shella. Shell čita datoteku i izvršava naredbu po naredbu kao da je pisana putem tipkovnice. Na taj način ne moramo pamtiti naredbe, poslovi se mogu izvršavati u određeno vrijeme, ne gubimo vrijeme rješavajući problem putem GUI-a.

\begin{primjer} Primjeri shell skripti: 
\begin{itemize}
\item  naći sve izvršne programe u PATH-u koji odgovaraju nekom \textit{patternu} \url{http://user.it.uu.se/~matkin/documents/shell/#Rename_files}
\item  napraviti \textit{backup} za sve datoteke koje su mijenjane u nekom danu \url{http://tldp.org/LDP/abs/html/special-chars.html#EX58}
\item  konvertirati jedan grafički format u drugi (JPEG u PNG) \url{http://en.wikipedia.org/wiki/Shell_script}
\item  obavijestiti kada se neki user ulogira ili odlogira \url{http://www.comp.eonworks.com/scripts/scripts.html}
\item ispisati imena datoteka koje u sebi sadrže liniju dužu od $n$ znakova \url{http://www.comp.eonworks.com/scripts/scripts.html}
\item jednom tjedno kopirati sve datoteke na "backup server" na lokalnoj mreži.
itd.
\end{itemize}
\end{primjer}

\subsection*{Pisanje shell skripte}
Koraci:
\begin{enumerate}
 \item Napisati skriptu npr. \texttt{script1} u nekom editoru (npr. \texttt{gedit}). U prvom redku napisati \lstinline!#!!/bin/bash.
\item Dodijeliti pravo izvršavanja (\texttt{x}) skripti npr. \lstinline!chmod 755 script1!
\item Smjestiti je negdje gdje je shell može naći npr. u tekući direktorij 
\item Izvršiti je npr. \texttt{./script1}
\end{enumerate}

\begin{zadatak}Napisati Hello world! skriptu i izvršiti je.
\begin{lstlisting}
#!/bin/bash 
echo "Hello world!"
\end{lstlisting}
\end{zadatak}
\subsection*{Elementi shell skripte}
\begin{itemize}
 \item \textbf{Varijable: } definiraju se ovako: \lstinline!X="Hello"!, a vrijednost se dohvaća sa \lstinline!$X!. U varijablu možemo spremiti i vrijednost nekog aritmetičkog izraza koristeći dvostruke okrugle zagrade npr.~\lstinline!X=$(($Y+$Z))!
\item \textbf{Standardne (environment) varijable: } su varijable shella (npr. \lstinline!$PATH!, \lstinline!$USER!, \lstinline!$PWD! ili \lstinline!$OLD_PWD!) koje možete pogledati naredbom \texttt{env}.
\item \textbf{Parametri tj. argumenti komandne linije} označeni su sa \lstinline!$1!, \lstinline!$2!, \lstinline!$3! itd.
\begin{itemize}
 \item \lstinline!$@! je \lstinline!"$1"! \lstinline!"$2"! \lstinline!"$3"!
\item \lstinline!$*! je \lstinline!"$1 $2 $3"!
\item \lstinline!$\#! je broj parametara
\end{itemize}

\end{itemize}

\begin{zadatak}
Napišite (i proučite) skriptu \texttt{prva} i izvršite je sa \texttt{./prva jen dva}. Što skripta radi?
\begin{lstlisting}
#!/bin/bash
echo Ime skripte: $0
echo Prvi parametar: $1
echo Drugi parametar: $2
echo Broj parametara: $#
\end{lstlisting}
\end{zadatak}
\subsection*{Praznine}
Znak jednakosti (\lstinline!=!) koristi se za pridjeljivanje vrijednosti i za usporedbu.
\begin{itemize}
\item 
Kod pridjeljivanja vrijednosti znak \lstinline!=! nije omeđen prazninama: \lstinline!val=33! ili \lstinline!str="Neki string"!.
\item Kod uspoređivanja je znak \lstinline!=! omeđen prazninama: \lstinline!if [ $str = "Nekistring" ]; then echo "je"; else echo "nije"; fi!
\end{itemize}

\subsection*{Jednostruki vs. dvostruki navodnici}
\paragraph{strong quoting} Unutar jednostrukih navodnika ništa se ne interpretira. Primjer: \lstinline!echo 'Your PATH is: $PATH' ! ispisati će se \lstinline!Your PATH is: $PATH!. 
Znakovi nemaju posebno značenje
\paragraph{weak quoting} Unutar dvostukih navodnika interpretira se većina znakova, ali neki znakovi imaju posebno značenje i njih se ne interpretira:
\begin{itemize}
 \item \textit{space} ne interpretira kao separator riječi npr.
 
\begin{itemize}
\item 
 \texttt{touch prva datoteka.dat} kreirati će (ili promijeniti vrijeme pristupa) dvije datoteke: \texttt{prva} i \texttt{datoteka.dat} 
 
\item  \texttt{touch "prva datoteka.dat"} kreirati će (ili promijeniti vrijeme pristupa) datoteku: \texttt{prva datoteka.dat}                                                                                                                       \end{itemize}


 \item znakove unutar jednostrukih navodnika ne interpretira kao "strong quoting" npr.
 
\lstinline!echo "'$PATH'"! će ispisati cijeli \textit{path} 
 \item wildcardove ne interpretira npr.
 
 \lstinline!ls -l "prva*"! će ispisati poruku: \texttt{ls: cannot access prva*: No such file or directory}
 \item ekspanziju pathnamea i varijabli
 
\lstinline!echo "Koristimo $TERM"! će ispisati \texttt{Koristimo xterm}, odnosno ime terminala kojeg koristimo.

\item zamjene za komandu (\textit{command substitution}) npr. 
\lstinline!echo "`ls`"! će izvršiti naredbu \texttt{ls} 
\end{itemize}
\begin{zadatak}
Napišite (i proučite) skriptu \texttt{druga} i izvršite je. Što skripta radi?
\begin{lstlisting}
#!/bin/bash
echo "hello, $USER"
echo "Sadrzaj trenutnog direktorija, $PWD"
ls -l # list files
\end{lstlisting}
\end{zadatak}


\begin{zadatak}
 Napišite (i proučite) skriptu \texttt{treca} i izvršite je. Što skripta radi?
\begin{lstlisting}
#!/bin/bash
X=3
Y=4
echo $(($X+$Y))
echo $(($X-$Y))
echo $(($X*$Y))
echo $(($X/$Y))
echo $(($X%$Y))

\end{lstlisting}
\end{zadatak}

\begin{zadatak}
Modificirajte skriptu tako da vrši osnovne aritmetičke operacije nad argumentima komandne linije.
\end{zadatak}

\begin{comment}

\vfill
\begin{itemize}
\renewcommand{\labelitemi}{\textbf{$\rightarrow$}}
\item Popis svih pokrenutih naredbi zajedno sa zadnjim zadatkom eksportirajte u datoteku imena \texttt{prezime\_ime\_vj8.txt}. Uploadajte datoteku na \href{https://moodle.oss.unist.hr/course/view.php?id=133}{http://moodle.oss.unist.hr}.
\end{itemize}

\end{comment}
