%\usepackage{etex}
%\reserveinserts{28}
%----------------------------------------------------------------------------------------
%	DOCUMENT CONFIGURATIONS
%----------------------------------------------------------------------------------------
\usepackage{geometry} % Page margins
\geometry{top=2cm,bottom=2cm,left=2.5cm,right=2.5cm}

\usepackage{xcolor} % Required for specifying colors by name
\definecolor{ocre}{RGB}{243,102,25} %
\definecolor{lightcyan}{rgb}{0.88,1,1}
\definecolor{lightblue}{rgb}{0.8,0.85,1}
\definecolor{gray}{gray}{0.6}
\definecolor{turquoise}{rgb}{0 0.41 0.41}
\definecolor{green}{rgb}{0,0.6,0}
\definecolor{strcol}{rgb}{0.80,0.20,0.20}

% Font Settings
\usepackage{avant} % Use the Avantgarde font for headings
%\usepackage{MyriadPro}
%\usepackage{mathptmx} % Use the Adobe Times Roman as the default text font together with math symbols from the Sym­bol, Chancery and Com­puter Modern fonts
\usepackage{microtype} % Slightly tweak font spacing for aesthetics
%\usepackage[urw-garamond]{mathdesign}
\usepackage{mathpazo}
\usepackage[utf8]{inputenc} % Required for including letters with accents
\usepackage[T1]{fontenc} % Use 8-bit encoding that has 256 glyphs

%\renewcommand{\familydefault}{\sfdefault} % cijeli fon sans serif

\usepackage[croatian]{babel} % Croatian language/hyphenation
\usepackage[full]{textcomp} %trebam radi \textquotesingle

%\usepackage{ltablex}%
\usepackage{tabularx}
%\usepackage{showframe} 
\usepackage{verbatim}%
\usepackage{upquote}%
%\usepackage{wrapfig}%
%\usepackage{subfig}%
%\usepackage{subcaption}
\usepackage{longtable}
\usepackage{caption}
\newcommand{\myparagraph}[1]{\paragraph{#1}\mbox{}\\}

\renewcommand{\labelitemii}{$\cdot$} %itemize drugi level
%\usepackage[toc,page]{appendix}

% Bibliography
%\usepackage[style=alphabetic,sorting=nyt,sortcites=true,autopunct=true,babel=hyphen,hyperref=true,abbreviate=false,backref=true,backend=biber]{biblatex}
%\addbibresource{PMA_biljeske_i_vjezbe.bib} % BibTeX bibliography file
%\defbibheading{bibempty}{}

% Index
%\usepackage{calc} % For simpler calculation - used for spacing the index letter headings correctly
%\usepackage{makeidx} % Required to make an index
%\makeindex % Tells LaTeX to create the files required for indexing

%----------------------------------------------------------------------------------------
%----------------------------------------------------------------------------------------
%	VARIOUS RrequQUIRED PACKAGES
%----------------------------------------------------------------------------------------

\usepackage{titlesec} % Allows customization of titles

\usepackage{graphicx} % Required for including pictures
\graphicspath{{./images/}} % Specifies the directory where pictures are stored
\usepackage{float}

\usepackage{tikz} % Required for drawing custom shapes


\usepackage{enumitem} % Customize lists
\setlist{nosep} % Reduce spacing between bullet points and numbered lists

\usepackage{booktabs} % Required for nicer horizontal rules in tables, \toprule, \midrule, and \bottomrule macros 

\usepackage{eso-pic} % Required for specifying an image background in the title page

%\usepackage{wrapfig}
%\usepackage{subfig}

\usepackage[breaklinks]{hyperref}
\usepackage{pdfpages} % za naslovnicu

%\usepackage{xcolor}
\definecolor{dark-red}{rgb}{0.4,0.15,0.15}
\definecolor{dark-blue}{rgb}{0.15,0.15,0.4}
\definecolor{medium-blue}{rgb}{0,0,0.5}
\hypersetup{
    colorlinks=true, pdfborder={0 0 0}, linkcolor={ocre},
    citecolor={ocre}, urlcolor={medium-blue},
    pdftitle = {Operativni sustavi},
    pdfauthor = {Ljiljana Despalatović}
}



%----------------------------------------------------------------------------------------
%	MAIN TABLE OF CONTENTS
%----------------------------------------------------------------------------------------

\usepackage{titletoc} % Required for manipulating the table of contents

\contentsmargin{0cm} % Removes the default margin
% Chapter text styling
\titlecontents{chapter}[1.25cm] % Indentation
{\addvspace{15pt}\large\sffamily\bfseries} % Spacing and font options for chapters
{\color{ocre!60}\contentslabel[\Large\thecontentslabel]{1.25cm}\color{ocre}} % Chapter number
{}  
{\color{ocre!60}\normalsize\sffamily\bfseries\;\titlerule*[.5pc]{.}\;\thecontentspage} % Page number
% Section text styling
\titlecontents{section}[2.25cm] % Indentation
{\addvspace{5pt}\sffamily\bfseries} % Spacing and font options for sections
{\contentslabel[\thecontentslabel]{1.25cm}} % Section number
{}
{\sffamily\hfill\color{black}\thecontentspage} % Page number
[]
% Subsection text styling
\titlecontents{subsection}[3.25cm] % Indentation
{\addvspace{1pt}\sffamily\small} % Spacing and font options for subsections
{\contentslabel[\thecontentslabel]{1.25cm}} % Subsection number
{}
{\sffamily\;\titlerule*[.5pc]{.}\;\thecontentspage} % Page number
[] 

%----------------------------------------------------------------------------------------
%	PAGE HEADERS
%----------------------------------------------------------------------------------------

\usepackage{fancyhdr} % Required for header and footer configuration

\pagestyle{fancy}
\renewcommand{\chaptermark}[1]{\markboth{\sffamily\normalsize\bfseries #1}{}} % Chapter text font settings
\renewcommand{\sectionmark}[1]{\markright{\sffamily\normalsize\thesection\hspace{5pt}#1}{}} % Section text font settings
\fancyhf{} \fancyhead[LE,RO]{\sffamily\normalsize\thepage} % Font setting for the page number in the header
\fancyhead[LO]{\rightmark} % Print the nearest section name on the left side of odd pages
\fancyhead[RE]{\leftmark} % Print the current chapter name on the right side of even pages
\renewcommand{\headrulewidth}{0.5pt} % Width of the rule under the header
\addtolength{\headheight}{2.5pt} % Increase the spacing around the header slightly
\renewcommand{\footrulewidth}{0pt} % Removes the rule in the footer
\fancypagestyle{plain}{\fancyhead{}\renewcommand{\headrulewidth}{0pt}} % Style for when a plain pagestyle is specified

% Removes the header from odd empty pages at the end of chapters
\makeatletter
\renewcommand{\cleardoublepage}{
\clearpage\ifodd\c@page\else
\hbox{}
\vspace*{\fill}
\thispagestyle{empty}
\newpage
\fi}

%----------------------------------------------------------------------------------------
%	THEOREM STYLES
%----------------------------------------------------------------------------------------

 
\usepackage{amsmath,amsfonts,amssymb,amsthm} % For including math equations, theorems, symbols, etc
\begin{comment}
\newcommand{\intoo}[2]{\mathopen{]}#1\,;#2\mathclose{[}}
\newcommand{\ud}{\mathop{\mathrm{{}d}}\mathopen{}}
\newcommand{\intff}[2]{\mathopen{[}#1\,;#2\mathclose{]}}
\newtheorem{notation}{Notation}[chapter]

\end{comment}
 
\newtheoremstyle{ocrenum} % Theorem style name
{20pt} % Space above
{10pt} % Space below
{\normalfont} % Body font
{} % Indent amount
{\small\bf\sffamily\color{ocre}} % Theorem head font
{\;\;} % Punctuation after theorem head
{0.25em} % Space after theorem head
{\small\sffamily\color{ocre}\thmname{#1}\thmnumber{\@ifnotempty{#1}{ }\@upn{#2}} % Theorem text (e.g. Theorem 2.1)
\thmnote{\ {\the\thm@notefont\sffamily\bfseries\color{black}--- #3.}}} % Optional theorem note
\renewcommand{\qedsymbol}{$\blacksquare$} % Optional qed square


% Defines the theorem text style for each type of theorem to one of the three styles above
\theoremstyle{ocrenum}
\newtheorem{problem}{Problem}[chapter]
\newtheorem{vjezba}{Vježba}[chapter]
\newtheorem{primjer}{Primjer}[chapter]
\newtheorem{zadatak}{Zadatak}[chapter]
\newtheorem{napomena}{Napomena}[chapter]


%----------------------------------------------------------------------------------------
%	CHAPTER HEADINGS
%----------------------------------------------------------------------------------------

\newcommand{\thechapterimage}{}
\newcommand{\chapterimage}[1]{\renewcommand{\thechapterimage}{#1}}
\def\thechapter{\arabic{chapter}}
\def\@makechapterhead#1{
\thispagestyle{empty}
{\centering \normalfont\sffamily
\ifnum \c@secnumdepth >\m@ne
\if@mainmatter
\startcontents
\begin{tikzpicture}[remember picture,overlay]
\node at (current page.north west)
{
  \begin{tikzpicture}[remember picture,overlay]
  \node[anchor=north west] at (-4pt,4pt) {\includegraphics[width=\paperwidth]{\thechapterimage}};
  \draw[anchor=west] (5cm,-8cm) node [rounded corners=25pt, fill=white, opacity=.7, draw=ocre, draw opacity=1, line width=2pt, inner sep=15.5pt]{\huge\sffamily\bfseries\textcolor{black}{#1\vphantom{plPQq}\makebox[7.5cm]{}}};
  \end{tikzpicture}};
\end{tikzpicture}}\par\vspace*{230\p@}
}
\def\@makeschapterhead#1{
\thispagestyle{empty}
{\centering \normalfont\sffamily
\ifnum \c@secnumdepth >\m@ne
\if@mainmatter
\startcontents

\begin{tikzpicture}[remember picture,overlay]
\node at (current page.north west)
{\begin{tikzpicture}[remember picture,overlay]
\node[anchor=north west] at (-4pt,4pt) {\includegraphics[width=\paperwidth]{\thechapterimage}};
\draw[anchor=west] (5cm,-8cm) node [rounded corners=25pt, draw=ocre, fill=white, opacity=.7, line width=2pt, inner sep=15pt]{\huge\sffamily\bfseries\textcolor{black}{#1\vphantom{plPQq}\makebox[10cm]{}}};
\end{tikzpicture}};
\end{tikzpicture}}\par\vspace*{230\p@}
}
\makeatother

%---------------------------------------------
% WATERMARK
%---------------------------------------------
%\usepackage{draftwatermark}
%\SetWatermarkFontSize{2cm}
%\SetWatermarkText{draft}
%\SetWatermarkScale{3}
%\SetWatermarkLightness{0.95}


%--------------------------------------------------
% IMAGES
%--------------------------------------------------
\makeatletter
\g@addto@macro\@floatboxreset\centering
\makeatother

%---------------------------------------------------
% LISTINGS
%---------------------------------------------------
%\newcommand*{\myfont}{\fontfamily{Courier New}\selectfont}
%\newfontfamily\listingsfontinline[Scale=0.8]{Courier New} 
%\newfontfamily{\examplefont}{Helvetica}
\usepackage{framed}
\usepackage{tikz}
\usepackage{listings}
\lstset{language=C,  keywordstyle=\color{blue!50!black}, columns=flexible, showstringspaces=false, escapechar=@,breaklines=true, lineskip={-1.5pt}, basicstyle=\normalsize\ttfamily, commentstyle=\color{gray},    stringstyle=\color{orange}}
\lstdefinestyle{numbers} {numbers=left, stepnumber=1, numberstyle=\tiny, numbersep=10pt}
\lstdefinestyle{zadatak} {language=C, basicstyle=\normalsize\ttfamily, keywordstyle=\color{blue}, columns=flexible, showstringspaces=false, breaklines=true, lineskip={-1.5pt}}

\usetikzlibrary{decorations}
\usetikzlibrary{calc}
%\pgfmathsetseed{1} % To have predictable results
% Define a background layer, in which the parchment shape is drawn
\pgfdeclarelayer{background}
\pgfsetlayers{background,main}

% define styles for the normal border and the torn border
\tikzset{
  normal border/.style={ocre, rounded corners={0.2cm}, line width={0.05cm}},
  torn border/.style={ocre, rounded corners={0.2cm}, dotted, line width={0.05cm}}}

% Macro to draw the shape behind the text, when it fits completly in the
% page
\def\parchmentframe#1{
\tikz{
  \node[inner sep=0.5em] (A) {#1};  % Draw the text of the node
  %\node[fill=gray!5](box)% Makebox is needed to take the frame added by listings into account
  %\makebox[0.98\textwidth][l]{\box\@tempboxa};
  \begin{pgfonlayer}{background}  % Draw the shape behind
  \draw[normal border] 
        ($(A.south east)-(0.9,0)$) -- ($(A.south west)+(0.6,0)$) -- 
        ($(A.north west)+(0.6,0)$) -- ($(A.north east)-(0.9,0)$) -- cycle;
  \end{pgfonlayer}
  }}

% Macro to draw the shape, when the text will continue in next page
\def\parchmentframetop#1{
\tikz{
  \node[inner sep=2em] (A) {#1};    % Draw the text of the node
  \begin{pgfonlayer}{background}    
  \draw[normal border]              % Draw the ``complete shape'' behind
        %(A.south east) -- 
        ($(A.south west)+(0.6,0)$) -- 
        ($(A.north west)+(0.6,0)$) -- ($(A.north east)-(0.9,0)$) -- ($(A.south east)-(0.9,0)$);
  \draw[torn border]                % Add the torn lower border
         ($(A.south east)-(0.9,0)$) -- ($(A.south west)+(0.6,0)$);
        %($(A.south west)+(0.6,0)$) -- ($(A.south east)+(0.6,0)$) -- cycle;
  \end{pgfonlayer}
  }}

% Macro to draw the shape, when the text continues from previous page
\def\parchmentframebottom#1{
\tikz{
  \node[inner sep=2em] (A) {#1};   % Draw the text of the node
  \begin{pgfonlayer}{background}   
  \draw[normal border]             % Draw the ``complete shape'' behind
        ($(A.north east)-(0.9,0)$) -- ($(A.south east)-(0.9,0)$) -- ($(A.south west)+(0.6,0)$) -- 
        ($(A.north west)+(0.6,0)$);
  \draw[torn border]               % Add the torn upper border
        ($(A.north east)-(0.9,0)$) -- ($(A.north west)+(0.6,0)$);
  \end{pgfonlayer}
  }}

% Macro to draw the shape, when both the text continues from previous page
% and it will continue in next page
\def\parchmentframemiddle#1{
\tikz{
  \node[inner sep=2em] (A) {#1};   % Draw the text of the node
  \begin{pgfonlayer}{background}   
  \draw[normal border]             % Draw the ``complete shape'' behind
        ($(A.south west)+(0.6,0)$) -- ($(A.north west)+(0.6,0)$);
  \draw[normal border]             % Draw the ``complete shape'' behind
        ($(A.north east)-(0.9,0)$) -- ($(A.south east)-(0.9,0)$);        
  \draw[torn border]               % Add the torn lower border
        ($(A.south east)-(0.9,0)$) -- ($(A.south west)+(0.6,0)$);
  \draw[torn border]               % Add the torn upper border
        ($(A.north east)-(0.9,0)$) -- ($(A.north west)+(0.6,0)$);
  \end{pgfonlayer}
  }}

\lstnewenvironment{prototip}[1][]{%
  \lstset{xleftmargin=20pt}
  \def\FrameCommand{\parchmentframe}%
  \def\FirstFrameCommand{\parchmentframetop}%
  \def\LastFrameCommand{\parchmentframebottom}%
  \def\MidFrameCommand{\parchmentframemiddle}%
  %\vskip\baselineskip
  \MakeFramed{\FrameRestore}
  }%
{\endMakeFramed}

\lstnewenvironment{framedcode}[1][]{%
  \lstset{xleftmargin=20pt}
  \def\FrameCommand{\parchmentframe}%
  \def\FirstFrameCommand{\parchmentframetop}%
  \def\LastFrameCommand{\parchmentframebottom}%
  \def\MidFrameCommand{\parchmentframemiddle}%
  %\vskip\baselineskip
  \MakeFramed{\FrameRestore}
  \noindent\tikz\node[inner sep=1ex, draw=ocre,fill=white, rounded corners={0.2cm},  line width={0.05cm}, 
          anchor=west, overlay] at (0em, 2em) {#1};
  }%
{\endMakeFramed}

\lstnewenvironment{framedcodeitem}[1][]{%
  \lstset{xleftmargin=-20pt}
  \def\FrameCommand{\parchmentframe}%
  \def\FirstFrameCommand{\parchmentframetop}%
  \def\LastFrameCommand{\parchmentframebottom}%
  \def\MidFrameCommand{\parchmentframemiddle}%
  %\vskip\baselineskip
  \MakeFramed{\FrameRestore}
  \noindent\tikz\node[inner sep=1ex, draw=ocre,fill=white, rounded corners={0.2cm},  line width={0.05cm}, 
          anchor=west, overlay] at (0em, 2em) {#1};
  }%
{\endMakeFramed}

\lstnewenvironment{framedcodenumber}[1][]{%
  \lstset{style=numbers, xleftmargin=20pt}
  \def\FrameCommand{\parchmentframe}%
  \def\FirstFrameCommand{\parchmentframetop}%
  \def\LastFrameCommand{\parchmentframebottom}%
  \def\MidFrameCommand{\parchmentframemiddle}%
  %\vskip\baselineskip
  \MakeFramed{\FrameRestore}
  \tikz\node[inner sep=1ex, draw=ocre,fill = white, rounded corners={0.2cm}, line width={0.05cm}, 
          anchor=west, overlay] at (0em, 2em) {#1};
  }%
{\endMakeFramed}


\lstnewenvironment{terminall}[1][]{%
  \lstset{style=numbers, xleftmargin=20pt}
  \def\FrameCommand{\parchmentframe}%
  \def\FirstFrameCommand{\parchmentframetop}%
  \def\LastFrameCommand{\parchmentframebottom}%
  \def\MidFrameCommand{\parchmentframemiddle}%
  %\vskip\baselineskip
  \tikzstyle{terminal} = [
    draw=white, text=white, font=courier, fill=black, very thick,
    rectangle, rounded corners, inner sep=2pt, inner ysep=8pt
   ]
   \tikzstyle{terminalTitle} = [
     fill=black, text=white, font=\ttfamily, draw=white
   ]
  \MakeFramed{\FrameRestore}
  \noindent\resizebox{\textwidth}{!}{ % This line fits the box to textwidth
  \tikz\node[inner sep=1ex, draw=ocre,fill = white, rounded corners={0.2cm}, line width={0.05cm}, 
          anchor=west, overlay] at (0em, 2em) {#1};
  }}%
  {\endMakeFramed}

%\usetikzlibrary{shapes}
%\lstset{basicstyle=\ttfamily\footnotesize,breaklines=true}
\newsavebox\terminalbox
\lstnewenvironment{terminal}[1][]
  {\lstset{#1}\setbox\terminalbox=\vbox\bgroup\hsize=0.8\textwidth}
  {\egroup
   \tikzstyle{terminal} = [
    draw=white, text=white, font=courier, fill=black, very thick,
    rectangle, rounded corners, inner sep=2pt, inner ysep=8pt
   ]
   \tikzstyle{terminalTitle} = [
     fill=black, text=white, font=\ttfamily, draw=white
   ]
   \noindent\resizebox{\textwidth}{!}{ % This line fits the box to textwidth
   \begin{tikzpicture}
   \node [terminal] (box){\usebox{\terminalbox}};
   \node[terminalTitle, right=10pt, rounded corners] at (box.north west) {tty: /bin/bash};
   \end{tikzpicture}}
}		


%-----------------------------------------
% TABLES
%-----------------------------------------
%\usepackage{ltablex}
\renewcommand{\arraystretch}{0.8} %more/less space between rows

%-----------------------------------------
% ALGORITHMS
%-----------------------------------------
\begin{comment}
 
\usepackage[ruled]{algorithm2e}
\usepackage{algpseudocode}

\SetAlgorithmName{Algoritam}{algoritam}{Lista algoritama}

\end{comment}



%-----------------------------------------
% VARIOUS
%-----------------------------------------
\usepackage{verbatim}
\usepackage{todonotes}
%\usepackage[disable]{todonotes}

\usepackage{bitset}
%\usepackage{fmtcount}% http://ctan.org/pkg/fmtcount
%\newcounter{mycounter}%
%\newcommand{\binnum}[2][4]{%
%  \setcounter{mycounter}{\numexpr #2\relax}%
%  \padzeroes[#1]{\binary{mycounter}}%
%}
%\usepackage{xparse}% http://ctan.org/pkg/xparse
%\NewDocumentCommand{\binnumm}{s O{4} m}{%
%  \setcounter{mycounter}{\numexpr #3\relax}%
%  \padzeroes[#2]{\binary{mycounter}}\IfBooleanTF{#1}{}{_2}%
%}

\setlength{\parskip}{6pt} %extra space between paragraphs

\newcommand{\ld}[1]{\textcolor[rgb]{1.00,0.50,0.00}
{\noindent\ensuremath{\langle}\textit{napomena: #1}\ensuremath{\rangle}}}
\newcommand{\bb}[1]{\textcolor[rgb]{1.00,0.00,0.00}
{\noindent\ensuremath{\langle}\textit{ljiljana: #1}\ensuremath{\rangle}}}

\author{Ljiljana Despalatovic}
\title{Operativni sustavi}
