\section*{Pipeline}

\textbf{Pipe} ili \textbf{pipeline} je niz od dvije ili više naredbi odvojenih vertikalnom crtom \lstinline!|!. Standardni izlat prve naredbe, standardni je ulaz druge naredbe.

\todo[inline]{Smislit prijevod za pipeline. Cjevovod, kanal?}

\begin{zadatak} Izvršite sljedeće linije
\begin{itemize}
\item \lstinline!ls | wc -l!
\item \lstinline!ls | grep dat!
\item \lstinline!ls | tac! - ispis u obratnom redoslijedu (pogledati \texttt{man tac})

%%\item U jednoj liniji ispišite datum i sve logirane usere i spremite u novu datoteku users nakon sto ste prebrojali linije 
%(date ; who) | nl > users
\end{itemize}
\end{zadatak}
\begin{zadatak} 
U direktoriju kreirajte direktorij \texttt{etc}.
\todo[inline]{Ja bih umjesto riječi kreirajte koristio napravite. Treba ujednačiti u cijeloj skripti}

\begin{enumerate}
\item U direktoriju \texttt{etc} kreirajte datoteku sa nastavkom \texttt{.txt} u kojoj su zapisane datoteke iz direktorija \texttt{/etc} koje počinju sa slovom \texttt{a} (\texttt{ls -d /etc/a*}). Uočite razliku između \texttt{etc} i \texttt{/etc}. Koristite preusmjeravanje \texttt{>} u datoteku. 
\item U direktoriju \texttt{etc} kreirajte datoteku sa nastavkom \texttt{.dat} u kojoj su zapisane datoteke iz direktorija \texttt{/etc} koje počinju sa slovom \texttt{a} u obratnom redoslijedu (\texttt{ls -dr /etc/a*}). 

%\begin{itemize}
\item Isprobajte: \lstinline!ls | grep dat!. Umjesto na terminal, upišite izlaz u datoteku \texttt{listing}.
\item Iz datoteke \texttt{/etc/passwd} izdvojite prvi i šesti stupac koji se odnose na korisnikaa \texttt{mail}. Kombinirajte naredbe \texttt{grep} i \texttt{cut}.
\item Izlistajte datoteke i direktorije u tekućem direktoriju sortirane po veličini. Kombinirajte \lstinline!ls -l! i \lstinline!sort -n -k 5! (sortiranje po petom stupcu).
\item Izlistajte datoteke (samo datoteke) u tekućem direktoriju sortirane po veličini. Za pronalaženje datoteka koristite \lstinline!find . -type f -ls!.

\item Iz datoteke \texttt{/etc/shadow} izdvojite stupac koji prikazuje enkriptiranu lozinku za korisnika \texttt{student} (lozinka se nalazi u drugom polju; prema potrebi koristite naredbu \texttt{sudo})

\item Snimite tri rečenice u tri različite datoteke naziva \texttt{prva}, \texttt{druga} i \texttt{treca}
\item Ispišite na terminal spojeni sadržaj sve tri datoteke \lstinline!cat prva druga treca!
\item Spojite sadržaj svih datoteka u datoteku naziva \texttt{prva\_druga\_treca}
\item Spojite sadržaj svih datoteka sortiran po abecedi u datoteku naziva \texttt{prva\_druga\_treca\_sorted}
%%\item U jednoj liniji ispišite datum i sve logirane usere i spremite u novu datoteku users nakon sto ste prebrojali linije 
%(date ; who) | nl > users
\end{enumerate}
\end{zadatak}
\section{Filteri}
\textbf{Filter} je naredba koja čita sa standardnog ulaza (\texttt{stdin}), trensformira ulaz i piše na standardni output. Može biti srednji dio pipelinea.

\begin{primjer} Isprobajte sljedeće naredbe
\begin{itemize}
\item \texttt{ls | grep prva | wc -l }
\item \texttt{cat etc/etc.dat etc/etc.txt | sort -n -k 5 | uniq > finaletc.list}  (\texttt{cat} spaja datoteke (konkatenacija), one se zatim sortiraju po petom polju i sa uniq se izbacuju duplikati, te se rezultat zapiše u \texttt{finaletc.list})
\end{itemize}
\end{primjer}

\begin{zadatak} 
Koristeci naredbu \texttt{grep} iz datoteke \texttt{/etc/passwd} izdvojite sve retke u kojima je ljuska \texttt{bash} (stupac 7). Ispišite samo prvi i sedmi stupac tj. \texttt{username} i \texttt{shell} (naredba \texttt{cut}) i sortirajte po korisničkom imenu (naredba \texttt{sort}). 
\end{zadatak}

\begin{comment}

\vfill
\begin{itemize}
\renewcommand{\labelitemi}{\textbf{$\rightarrow$}}
\item Popis svih pokrenutih naredbi eksportirajte u datoteku imena \texttt{prezime\_ime\_vj6.txt}. Uploadajte datoteku na \href{https://moodle.oss.unist.hr/course/view.php?id=133}{http://moodle.oss.unist.hr}.
\end{itemize}

\end{comment}


