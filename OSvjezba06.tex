\begin{zadatak} 
U direktoriju kreirajte direktorij \texttt{etc}.

\begin{enumerate}
\item U direktoriju \texttt{etc} kreirajte datoteku sa nastavkom \texttt{.txt} u kojoj su zapisane datoteke iz direktorija \texttt{/etc} koje počinju sa slovom \texttt{a} (\texttt{ls -d /etc/a*}). Uočite razliku između \texttt{etc} i \texttt{/etc}. Koristite preusmjeravanje \texttt{>} u datoteku. 
\item U direktoriju \texttt{etc} kreirajte datoteku sa nastavkom \texttt{.dat} u kojoj su zapisane datoteke iz direktorija \texttt{/etc} koje počinju sa slovom \texttt{a} u obratnom redoslijedu (\texttt{ls -dr /etc/a*}). 
\end{enumerate}
\end{zadatak}

\section*{Pipeline - nastavak}
\textbf{Pipe} ili \textbf{pipeline} je niz od dvije ili više naredbi odvojenih vertikalnom crtom \texttt{|}. Standardni output prve naredbe, standardni je input druge naredbe.
\begin{itemize}
\item \lstinline!ls | grep dat!
\item Iz datoteke \texttt{/etc/passwd} izdvojite prvi i šesti stupac koji se odnose na user-a \texttt{mail}. Kombinirajte \texttt{grep} i \texttt{cut}. 
\item Izlistajte datoteke i direktorije u tekućem direktoriju sortirane po veličini. Kombinirajte \lstinline!ls -l! i \lstinline!sort -n -k 5! (sortiranje po petom stupcu).
\item Izlistajte datoteke (samo datoteke) u tekućem direktoriju sortirane po veličini. Za pronalaženje datoteka koristite \lstinline!find . -type f -ls!.
% grep mail /etc/passwd | cut -f 1,6 -d :
%\item Iz datoteke /etc/shadow izdvojite stupac koji prikazuje enkriptirani password usera. (\texttt{sudo})
%\item Kombinirajte 2 datoteke u jednu cat *.dat > sve.dat
%%\item U jednoj liniji ispišite datum i sve logirane usere i spremite u novu datoteku users nakon sto ste prebrojali linije 
%(date ; who) | nl > users
\end{itemize}
\section*{Filteri}
\textbf{Filter} je naredba koja čita sa standardnog ulaza (\texttt{stdin}), trensformira ulaz i piše na standardni output. Može biti srednji dio pipelinea.

\begin{primjer} Isprobajte sljedeće naredbe
\begin{itemize}
\item \texttt{ls | grep .txt | wc -l }
\item \texttt{cat etc/etc.dat etc/etc.txt | sort -n -k 5 | uniq > finaletc.list}  (\texttt{cat} spaja datoteke (konkatenacija), 
one se zatim sortiraju po petom polju i sa uniq se izbacuju duplikati, te se rezultat zapiše u finaletc.list)
\end{itemize}
\end{primjer}

\begin{zadatak} 
Iz datoteke \texttt{/etc/passwd} izdvojite sve retke u kojima je shell \texttt{bash} (stupac 7) (naredba \texttt{grep}). Ispišite samo prvi i sedmi stupac tj. \texttt{username} i \texttt{shell} (naredba \texttt{cut}) i sortirajte po username-u (naredba \texttt{sort}). 
\end{zadatak}

\begin{comment}

\vfill
\begin{itemize}
\renewcommand{\labelitemi}{\textbf{$\rightarrow$}}
\item Popis svih pokrenutih naredbi eksportirajte u datoteku imena \texttt{prezime\_ime\_vj6.txt}. Uploadajte datoteku na \href{https://moodle.oss.unist.hr/course/view.php?id=133}{http://moodle.oss.unist.hr}.
\end{itemize}

\end{comment}


