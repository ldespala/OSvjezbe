\section{Shell skripte ... nastavak}

\subsection*{Uvjetno izvršavanje}
\textbf{Uvjetno izvršavanje} je izvršavanje naredbe ili skupa naredbi u ovisnosti o povratnoj vrijednosti druge naredbe. Realizira se pomoću \texttt{if then else fi} notacije.

\paragraph{Sintaksa \texttt{if} naredbe}
\begin{lstlisting}
if test-commands; then
    consequent-commands;
[elif more-test-commands; then
    more-consequents;]
[else alternate-consequents;]
fi
\end{lstlisting}

Ukoliko se izvršava samo jedna naredba moguće je zamijeniti \texttt{if then fi} sa naredbama \texttt{\&\&} i \texttt{||}.
\begin{itemize}
 \item 
\lstinline!command1 && command2 ! je ekvivalentno 

\lstinline!if (command1 successful) then do (command2)!

\item \lstinline!command1 || command2 ! je ekvivalentno 

\lstinline!if (command1 not successful) then do (command2)!

\end{itemize}


\subsection*{Uglate zagrade}
% http://wiki.bash-hackers.org/commands/classictest
Uglate zagrade su samo pokrata za naredbu \texttt{test}. 
\\
Sintaksa: \texttt{test <IZRAZ>}
\\
\texttt{<IZRAZ>} je kombinacija opcija i vrijednosti.
\begin{lstlisting}
#!/bin/bash
# test da li /etc/passwd postoji

if test -e /etc/passwd; then
  echo "Alright man..." >&2
else
  echo "Yuck! Where is it??" >&2
  exit 1
fi
\end{lstlisting}
Drugi način je testiranje pomoću uglate zagrade \texttt{[}. Naredba završava zatvorenom uglatom zagradom \texttt{]}. \\
Uglata zagrada \texttt{[} nije dio \texttt{if} naredbe.\\.
Sintaksa je \texttt{[ <IZRAZ> ]} (uočite razmak nakon otvorene uglate zagrade).
\begin{lstlisting}
#!/bin/bash
# test da li /etc/passwd postoji

if [ -e /etc/passwd ]; then
  echo "Alright man..." >&2
else
  echo "Yuck! Where is it??" >&2
  exit 1
fi
\end{lstlisting}

Za \texttt{test-commands} pogledajte link \href{http://www.ibm.com/developerworks/library/l-bash-test/index.html}{Bash test and comparison functions}. Neki od test naredbi su dani u 
sljedećoj tablici:\\
\begin{longtable}{ll}
  \hline
  \multicolumn{2}{c}{Izrazi za testiranje statusa datoteke} \\
  \hline
  \texttt{-f <file>} & file je obična datoteka \\
  \texttt{-d <file>} & file je direktorij \\
  \texttt{-r <file>} & file ima pravo čitanja \\
  \texttt{-w <file>} & file ima pravo pisanja \\
  \texttt{-x <file>} & file ima pravo izvršavanja \\
  \texttt{-s <file>} & file nije prazan (ima nenula duljinu) \\
  \texttt{-e <file>} & file postoji \\
  \hline
  \multicolumn{2}{c}{Izrazi za testiranje stringova} \\
  \hline
  \texttt{-n <string>} & string nije prazan \\
  \texttt{-z <string>} & string je prazan \\
  \texttt{<string> == <string>} & stringovi su jednaki \\
\texttt{<string> != <string>} & stringovi nisu jednaki \\
\hline
  \multicolumn{2}{c}{Aritmetički izrazi} \\
  \hline
  \texttt{<value> -eq <value>} & jednakost \\
  \texttt{<value> -ne <value>} & nejednakost \\
  \texttt{<value> -lt <value>} & manje od \\
  \texttt{<value> -le <value>} & manje ili jednako \\
  \texttt{<value> -gt <value>} & veće od \\
  \texttt{<value> -ge <value>} & veće ili jednako \\
\hline
\end{longtable}
\begin{primjer} 
Izvršite sljedeću skriptu. Što radi?
\begin{lstlisting}
#!/bin/bash
A=101
if [ "$A" -eq 10 ]
then
    echo A = 10
else
    echo A != 10
fi
\end{lstlisting}
\end{primjer}
\begin{zadatak} Modificirajte zadatak iz prethodne vježbe (osnovne aritmetičke operacije s parametrima) tako da se parametri zbroje ako je prvi parametar manji od drugog parametra, 
pomnože ako su jednaki, a oduzmu ako je prvi parametar veći od drugog parametra.
\end{zadatak}
\subsection*{Naredbe za ponavljanje}

 \subsubsection*{Naredba \texttt{while}}
Sintaksa:
\begin{lstlisting}
while expression is true
do
    command(s)
done
\end{lstlisting}

\begin{primjer} Proučite i izvršite!
\begin{lstlisting}
#!/bin/bash
count=0
max=10
while [ $count != $max ]; do count=`expr $count + 1`
    echo "$count"
done
\end{lstlisting}
\end{primjer}

\subsubsection*{Naredba \texttt{for}}
Sintaksa:
\begin{lstlisting}
for identifier in list
do
    command(s) to be executed
done
\end{lstlisting}

\begin{primjer} Proučite i izvršite!
\begin{enumerate}
 \item
\begin{lstlisting}
#!/bin/bash
for i in 1 2 3 4 
do
   echo "$i"
done 
\end{lstlisting}

\item 
\begin{lstlisting}
#!/bin/bash
for i in {1..4}
do
   echo "$i"
done 
\end{lstlisting}

\item 
\begin{lstlisting}
#!/bin/bash
for i in {1..10..2}
do
   echo "$i"
done 
\end{lstlisting}

\item 
\begin{lstlisting}
#!/bin/bash
for (( i=0; i<5; i++ ))
do
   echo "$i"
done 
\end{lstlisting}

\item 
\begin{lstlisting}
#!/bin/bash
count=0
for d in *
do
  count=`expr $count + 1`
  echo $d
done
echo $count
\end{lstlisting}
\end{enumerate}
\end{primjer}


\begin{comment}

\vfill
\begin{itemize}
\renewcommand{\labelitemi}{\textbf{$\rightarrow$}}
\item Popis svih pokrenutih naredbi zajedno sa skriptama eksportirajte u datoteku imena \texttt{prezime\_ime\_vj9.txt}. Uploadajte datoteku na \href{https://moodle.oss.unist.hr/course/view.php?id=133}{http://moodle.oss.unist.hr}.
\end{itemize}

\end{comment}
