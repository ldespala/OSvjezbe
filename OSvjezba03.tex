\newcommand{\sitosig}[2]{\textcolor{dark-blue}{\texttt{/#1}\texttt{\ ::\ }{\texttt{#2}/}}}

\section{Naredbe za navigaciju datotečnim sustavom}
U radu s terminalom bitan podatak je tekući direktorij. To je trenutna pozicija u datotečnom sustavu.
Datotečni sustav ima hijerarhijsku strukturu direktorija. Na vrhu je root direktorij (/), njegova djeca su \texttt{bin}, \texttt{home}, \texttt{sbin}, \texttt{var}, \texttt{proc} itd.
\begin{itemize}
\item Naredba \texttt{pwd} ispisuje radni ili tekući direktorij. Prilikom pokretanja ljuske (eng. shell), tekući direktorij biti će \texttt{/home/student}.
\item \texttt{cd} - promjena tekućeg direktorija (\textit{change directiory})
\begin{prototip}
cd path
\end{prototip}

\begin{primjer} Korištenje naredbe \texttt{cd}.
\begin{itemize}
\item \texttt{cd naziv\_dir} - naziv\_dir postaje tekući direktorij
\item \texttt{cd ..} - prebacivanje u roditeljski direktorij
\item \texttt{cd /var/spool/anacron} -  \texttt{/var/spool/anacron} postaje tekući direktorij
\item \texttt{cd \~} prebacivanje u korisnikov \texttt{home} direktorij
\end{itemize}
\end{primjer}

\item Ljuska proširuje znak tilda u putanju do korisnikovog home direktorija

\begin{prototip}
echo ~
\end{prototip}

\item \texttt{ls} - daje popis datoteka i direktorija u danom direktoriju (ako nema parametara, izlista se tekući direktorij)
\begin{prototip}
ls direktorij
\end{prototip}
\begin{primjer} Primjer korištenja naredbe \texttt{ls}.
\begin{itemize}
 \item \texttt{ls naziv\_dir} - sadržaj direktorija \texttt{naziv\_dir}
\item \texttt{ls ..} - sadržaj roditeljskog direktorija
\item \texttt{ls /var} - sadržaj direktorija \texttt{/var}
\end{itemize}
 \textit{Opcije}
\newline Neke od opcija naredbe \texttt{ls}
\begin{itemize}
\item \texttt{ls -l} prikazuje detaljniji ispis
\item \texttt{ls -a} prikazuje i datoteke koje počinju s . (hidden file)
\item \texttt{ls -al} kombinacija prethodne dvije opcije
\item \texttt{ls -r} ispisuje sadržaj direktorija u obrnutom redoslijedu (leksikografski)
\end{itemize}
\end{primjer}

\item \texttt{mkdir} kreira direktorij
\begin{prototip}
mkdir naziv_dir 
\end{prototip}

\item \texttt{rmdir} briše direktorij. Direktorij treba biti prazan, inače \texttt{rm -r}
\begin{prototip}
rmdir naziv_dir 
\end{prototip}


\item \texttt{cp} kopira datoteke ili direktorije
\begin{prototip}
cp sto gdje 
\end{prototip}

\begin{primjer} Primjer korištenja naredbe \texttt{cp}
\begin{itemize}
 \item \lstinline!cp naziv.txt ~/dir1/! - kopira datoteku \texttt{naziv.txt} iz tekućeg direktorija u direktorij \lstinline!~/dir1/!
\item \lstinline!cp naziv.txt ../! - kopira datoteku \texttt{naziv.txt} iz tekućeg direktorija u roditeljski direktorij \texttt{..}
\item \lstinline!cp naziv.txt ../naziv2.txt! - kopira datoteku \texttt{naziv.txt} iz tekućeg direktorija u roditeljski direktorij \texttt{..} s novim nazivom.
 \item \lstinline!cp -R dir1 dir2! - kopira direktorij i sve njegove elemente u drugi direktorij 
\end{itemize}
\end{primjer}

\item \texttt{mv} premješta datoteke ili direktorije
\begin{prototip}
mv sto gdje
\end{prototip}

\begin{primjer} Primjer korištenja naredbe \texttt{mv}.
\begin{itemize}
 \item \lstinline!mv naziv.txt ~/dir1/! - premješta datoteku \texttt{naziv.txt} iz tekućeg direktorija u direktorij \lstinline!~/dir1/!
 \item \lstinline!mv -R dir1 dir2! - premješta direktorij i sve njegove elemente u drugi direktorij 
\item \lstinline!mv naziv.txt ../naziv2.txt! - premješta datoteku \texttt{naziv.txt} iz tekućeg direktorija u roditeljski direktorij \texttt{..} s novim nazivom.
 \item \lstinline!mv -R dir1 dir2! - premješta direktorij i sve njegove elemente u drugi direktorij 
\end{itemize}
\end{primjer}

\end{itemize}
%\section{Zadaci}
\begin{zadatak}
\begin{itemize}
\item Isprobajte što rade sljedeće naredbe: \textbf{\texttt{cd}}, \textbf{\texttt{ls}}, \textbf{\texttt{mkdir}}, \textbf{\texttt{rmdir}}, \textbf{\texttt{pwd}}.
\item Koristeći naredbu \textbf{\texttt{touch}} kreirati novu datoteku.
\item Sadržaj nekog direktorija ispišite u datoteku \texttt{listing} koristeći preusmjeravanje\textbf{ \texttt{>}}.
\item Koristeći naredbe \textbf{\texttt{cat}} i \textbf{\texttt{less}} prikazati sadržaj neke datoteke.
\end{itemize}

\end{zadatak}

\begin{zadatak} \sitosig{myfile}{B20421}
	\item Napravite praznu datoteku naziva \texttt{myfile}. Rezultat zadatka provjerite provlačenjem datoteke kroz \texttt{sito}.
\end{zadatak}

\noindent Program \texttt{sito} generira jedinstven potpis datoteke ili direktorija. U nekim od zadataka nalazit će se oznaka datoteke koju treba provjeriti i njen potpis, npr. \sitosig{myfile}{B20421}. Rješenje prethodnog zadatka provjerava se na sljedeći način:

\begin{prototip}
	sito myfile
\end{prototip}

\begin{zadatak} Riješite sljedeće zadatke koristeći naredbe \textbf{\texttt{cd}}, \textbf{\texttt{ls}}, \textbf{\texttt{mkdir}}, \textbf{\texttt{rmdir}}, \textbf{\texttt{mv}}, \textbf{\texttt{cp}}, \textbf{\texttt{pwd}}:

\begin{itemize}
\item Ispitajte koji je tekući direktorij (direktorij u kojem se nalazite). Naredba \texttt{pwd}.
\item Promjenite tekući direktorij u root (/). Naredba \texttt{cd}.
\item Ispišite putanju do svog home direktorija. Naredba \texttt{echo \$HOME}.
\end{itemize}
\end{zadatak}

\begin{zadatak} \sitosig{7E5AC6}{mydir}
\begin{itemize} 
	\item Prebacite se u svoj home direktorij.
	\item Napravite direktorij \texttt{mydir}  i prebacite se u njega.
	\item Napravite podirektorij \texttt{tmp} i u njemu datoteku \texttt{prva}. 
\end{itemize}
Rješenja zadatka u kojima je potpis napisan u zaglavlju provjeravaju se nakon što su napravljeni svi pojedinačni koraci u zadatku.
\end{zadatak}

\begin{zadatak}
Prebacite se u svoj home direktorij.
\begin{itemize}
\item Iscrtajte strukturu direktorija naredbom \texttt{tree}.
\item Izlistajte popis datoteka u tekućem direktoriju. Naredba \texttt{ls}.
\item Ispitajte opcije naredbe \texttt{ls}: \texttt{ls -a, ls -l, ls -d, ls -t, ls .., ls .} i kombinacije npr. \texttt{ls -ltr}.
\item Kopirajte datoteku \texttt{myfile} u novi direktorij \texttt{copies}. Naredba \texttt{cp}. \sitosig{EACF7B}{copies}
\item Napravite novu datoteku \texttt{test} i preimenujte je u \texttt{test.bak} koristeći naredbu \texttt{mv}. \sitosig{8F86AC}{test.bak}
\item Pomaknite datoteku \texttt{test.bak} u direktorij \texttt{copies}.
\item Izlistajte sadržaj direktorija \texttt{/usr} u obrnutom redoslijedu.
\item Kopirajte direktorij \texttt{mydir} i sve datoteke u njemu u direktorij \texttt{mydir2}. Koristite opciju \texttt{cp -R}.
\end{itemize}
\end{zadatak}

\section{Putevi \textit{paths}}
\begin{itemize}
 \item Prikažite svoj \texttt{search path}. Naredba \texttt{echo \$PATH\$} 
 
 \todo[inline]{Možda bi trebalo objasniti što je path. Koristio bih znak \$ samo prije naziva varijable, ne poslije. Ne znam je li pojam "search" path najbolji.}
 
%\item Eksportirajte neki path sa \texttt{export PATH=blabla} i pokušajte izlistati sadržaj direktorija.
\item Koja je putanja do vašeg \texttt{home} direktorija? Naredba \texttt{echo \~}. 
\item Idite u \texttt{/var/tmp} direktorij.
\item Idite u \texttt{/var/spool/anacron} direktorij koristeći samo jednu naredbu. Provjerite koji je vaš tekući direktorij?
\item Vratite se u \textit{home} direktorij.
\end{itemize}


% \section{Direktorij \texttt{/proc} }
% \begin{itemize}
%  \item Idite u \texttt{/proc} direktorij.
% \item Koji CPU koristi vaše računalo?
% \item Koliko RAM-a trenutno koristi?
% \item Koliki je \texttt{swap space}?
% \item Koje datotečne sustave poznaje vaš operativni sustav?
% \end{itemize}

\vfill
