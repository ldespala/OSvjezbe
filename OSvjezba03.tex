\section{Naredbe za navigaciju file systemom}
U radu s terminalom bitan podatak je tekući direktorij. To je trenutna pozicija u datotečnom sustavu.
Datotečni sustav ima hijerarhijsku strukturu direktorija. Na vrhu je root direktorij (/), njegova djeca su \texttt{bin}, \texttt{home}, \texttt{sbin}, \texttt{var}, \texttt{proc} itd.
\begin{itemize}
\item Naredba \texttt{pwd} ispisuje radni ili tekući direktorij. Prilikom pokretanja shella, tekući direktorij biti će \texttt{/home/student}.
\item \texttt{cd} - promjena tekućeg direktorija (\textit{change directiory})
\begin{prototip}
cd path
\end{prototip}

\begin{primjer} Korištenje naredbe \texttt{cd}.
\begin{itemize}
\item \texttt{cd ime\_dir} - tekući direktorij postaje ime\_dir
\item \texttt{cd ..} - tekući direktorij postaje roditelj
\item \texttt{cd /var/spool/anacron} - tekući direktorij postaje \texttt{/var/spool/anacron}
\end{itemize}
\end{primjer}

\item \texttt{ls} - daje popis datoteka i direktorija u danom direktoriju (ako nema parametara, izlista se tekući direktorij)
\begin{prototip}
ls direktorij
\end{prototip}
\begin{primjer} Primjer korištenja naredbe \texttt{ls}.
\begin{itemize}
 \item \texttt{ls ime\_dir} - sadržaj direktorija \texttt{ime\_dir}
\item \texttt{ls ..} - sadržaj roditeljskog direktorija
\item \texttt{ls /var} - sadržaj direktorija \texttt{/var}
\end{itemize}
 \textit{Opcije}
\newline Neke od opcija naredbe \texttt{ls}
\begin{itemize}
\item \texttt{ls -l} prikazuje detaljniji ispis
 \item \texttt{ls -a} prikazuje i datoteke koje počinju sa . (hidden file)
\item \texttt{ls -al} kombinacija prethodne dvije opcije
\item \texttt{ls -r} ispisuje sadržaj direktorija u obrnutom redoslijedu (leksikografski)
\end{itemize}
\end{primjer}

\item \texttt{mkdir} kreira direktorij
\begin{prototip}
mkdir ime_dir 
\end{prototip}

\item \texttt{rmdir} briše direktorij. Direktorij treba biti prazan, inače \texttt{rm -r}
\begin{prototip}
rmdir ime_dir 
\end{prototip}


\item \texttt{cp} kopira datoteke ili direktorije
\begin{prototip}
cp sto gdje 
\end{prototip}

\begin{primjer} Primjer korištenja naredbe \texttt{cp}
\begin{itemize}
 \item \lstinline!cp ime.txt ~/dir1/! - kopira datoteku \texttt{ime.txt} iz tekućeg direktorija u direktorij \lstinline!~/dir1/!
\item \lstinline!cp ime.txt ../! - kopira datoteku \texttt{ime.txt} iz tekućeg direktorija u roditeljski direktorij \texttt{..}
\item \lstinline!cp ime.txt ../ime2.txt! - kopira datoteku \texttt{ime.txt} iz tekućeg direktorija u roditeljski direktorij \texttt{..} s novim imenom.
 \item \lstinline!cp -R dir1 dir2! - kopira direktorij i sve njegove elemente u drugi direktorij 
\end{itemize}
\end{primjer}

\item \texttt{mv} premješta datoteke ili direktorije
\begin{prototip}
mv sto gdje
\end{prototip}

\begin{primjer} Primjer korištenja naredbe \texttt{mv}.
\begin{itemize}
 \item \lstinline!mv ime.txt ~/dir1/! - premješta datoteku \texttt{ime.txt} iz tekućeg direktorija u direktorij \lstinline!~/dir1/!
 \item \lstinline!mv -R dir1 dir2! - premješta direktorij i sve njegove elemente u drugi direktorij 
\item \lstinline!mv ime.txt ../ime2.txt! - premješta datoteku \texttt{ime.txt} iz tekućeg direktorija u roditeljski direktorij \texttt{..} s novim imenom.
 \item \lstinline!mv -R dir1 dir2! - premješta direktorij i sve njegove elemente u drugi direktorij 
\end{itemize}
\end{primjer}

\end{itemize}
%\section{Zadaci}
\begin{zadatak}
Isprobajte:
\begin{itemize}
\item Što rade sljedeće naredbe: \textbf{\texttt{cd}}, \textbf{\texttt{ls}}, \textbf{\texttt{mkdir}}, \textbf{\texttt{rmdir}}, \textbf{\texttt{pwd}}?
\item Koristeći naredbu \textbf{\texttt{touch}} kreirati novu datoteku.
\item Sadržaj nekog direktorija ispišite u novu datoteku koristeći preusmjeravanje\textbf{ \texttt{>}}.
\item Koristeći naredbe \textbf{\texttt{cat}} i \textbf{\texttt{less}} prikazati sadržaj neke datoteke.
\end{itemize}

\end{zadatak}

\begin{zadatak} Riješite sljedeće zadatke koristeći naredbe \textbf{\texttt{cd}}, \textbf{\texttt{ls}}, \textbf{\texttt{mkdir}}, \textbf{\texttt{rmdir}}, \textbf{\texttt{mv}}, \textbf{\texttt{cp}}, \textbf{\texttt{pwd}}:
\begin{itemize}
\item Ispitajte koji je tekući direktorij (direktorij u kojem se nalazite). Naredba \texttt{pwd}.
\item Promjenite tekući direktorij. Neka novi bude novokreirani direktorij. Naredba \texttt{cd}.
\item Kreirajte direktorij \texttt{tmp} i u njemu nekoliko datoteka (npr. imena prva, druga i treca).
\item Naredbom \texttt{tree} provjerite učinjeno.
\item Izlistajte popis datoteka u tekućem direktoriju. Naredba \texttt{ls}. Ispitajte opcije naredbe \texttt{ls}: \texttt{ls -a, ls -l, ls -d, ls -t, ls .., ls .} i kombinacije npr. \texttt{ls -ltr}.
\item Kopirajte datoteku \texttt{prva} u direktorij \texttt{vjezba3}. Naredba \texttt{cp}.
\item Preimenujte datoteku \texttt{prva} u \texttt{prvatmp} koristeći naredbu \texttt{mv}.
\item Pomaknite datoteku \texttt{druga} u direktorij \texttt{vjezba3} i preimenujte je koristeći samo jednu naredbu \texttt{mv}.
\item Izlistajte sadržaj direktorija \texttt{tmp} u obrnutom redoslijedu.
\item Kopirajte direktorij \texttt{tmp} i sve datoteke u njemu u direktorij \texttt{tmp2}. Koristite opciju \texttt{cp -R}.
\end{itemize}
\end{zadatak}
\section{Putevi \textit{paths}}
\begin{itemize}
 \item Prikažite svoj \texttt{search path}. Naredba \texttt{echo \$PATH\$}
%\item Eksportirajte neki path sa \texttt{export PATH=blabla} i pokušajte izlistati sadržaj direktorija.
\item Koja je putanja do vašeg \texttt{home} direktorija. Naredba \texttt{pwd}. 
\item Idite u \texttt{/var/tmp} direktorij.
\item Idite u \texttt{/var/spool/anacron} direktorij koristeći samo jednu naredbu. Koji je vaš tekući direktorij (\texttt{pwd})?
\item Vratite se u direktorij vjezba3.
\end{itemize}

\section{Direktorij \texttt{/proc} }
\begin{itemize}
 \item Idite u \texttt{/proc} direktorij.
\item Koji CPU koristi vaše računalo?
\item Koliko RAM-a trenutno koristi?
\item Koliki je \texttt{swap space}?
\item Koje file sisteme poznaje vaš operativni sustav?
\end{itemize}

%\textbf{File permissions}
%\begin{itemize}
%\item Možete li promijeniti datotečne dozvole \textit{file permissions} na \texttt{/home} direktoriju?
%\item Koji je standardni mod prilikom kreiranja datoteke?
%\item Promijenite dozvole za datoteke koje ste kreirali u direktoriju \texttt{tmp2}
%\end{itemize}
\vfill
