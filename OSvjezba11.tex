
\subsection*{Rad na udaljenom računalu}

Ova vježba radi se na udaljenom poslužitelju, koristeći program Putty koji komunicira s poslužiteljem na adresi 192.168.4.4 koristeći SSH protokol.

\subsection*{Spajanje na poslužitelj}
Prilikom spajanja na poslužitelj, studenti se spajaju koristeći korisničko ime i lozinku koje dobiju od nastavnika. 

\subsection*{Rad na poslužitelju}
Svaki korisnik na poslužitelju ima svoj direktorij (\texttt{/home/username}, odnosno \textasciitilde) i sve radnje treba obavljati unutar njega. Student ne bi trebao napuštati svoj direktorij, a u niti jednoj situaciji nije dozvoljeno pristupati tuđim podatcima. Općenito nije dozvoljeno na bilo koji način ugrožavati sigurnost ili remetiti rad sustava, te pristupati datotekama izvan vlastitog direktorija ili direktorija \texttt{/materijali}.

\begin{primjer}
	Pozicioniranje u \textit{home} direktorij
	\begin{itemize}
		\item \lstinline!cd ~!
		\item \lstinline!pwd!
		\item \lstinline!ls!
		\item \lstinline!ls ~!
		\item \lstinline!ls /home/102! 
	\end{itemize}
\end{primjer}

\begin{zadatak} Unutar direktorija \texttt{\textasciitilde} napravite poddirektorij \texttt{zad1} i u njemu riješite sljedeće zadatke:
	\begin{itemize}
		\item Ispitajte koji je tekući direktorij.
		\item Napravite direktorij \textasciitilde \texttt{/zad1/tmp}
		\item Promjenite tekući direktorij u novokreirani direktorij.
		\item u direktoriju \texttt{tmp} napravite datoteke \texttt{prva} \texttt{druga} \texttt{treca}.
		\item Isprobajte naredbu \texttt{tree} i njen izlaz preusmjerite u datoteku \texttt{treeoutput}
		\item Kopirajte datoteku \texttt{prva} u direktorij  \texttt{\textasciitilde}
		\item Preimenujte datoteku \texttt{prva} u \texttt{prvatmp}
		\item Pomaknite datoteku \texttt{druga} u direktorij  \texttt{\textasciitilde} i preimenujte je u \texttt{druga2} koristeći samo jednu naredbu \texttt{mv}.
		\item Kopirajte direktorij \texttt{zad1} i sve datoteke u njemu u direktorij \texttt{zad1b}.
	\end{itemize}
\end{zadatak}


\begin{zadatak} 
	U direktoriju \texttt{\textasciitilde} napravite direktorij \texttt{zad2} i uđite u njega.
		\item U direktoriju \texttt{zad2} kreirajte datoteku sa nastavkom \texttt{.txt} u kojoj su zapisane datoteke iz direktorija \texttt{/etc} koje počinju sa slovom \texttt{a}.
		
\end{zadatak}

\begin{zadatak}
	U direktoriju \texttt{\textasciitilde} napravite direktorij \texttt{zad3}  i uđite u njega. Datoteku \texttt{/materijali /mjesta} kopirajte u tekući direktorij. Datoteci \texttt{mjesta} dodijelite takva prava da samo vlasnik može čitati sadržaj datoteke.
\end{zadatak}

\begin{zadatak}
	U direktoriju \texttt{\textasciitilde} napravite direktorij \texttt{zad4} i uđite u njega. Datoteku \texttt{/materijali /mjesta} kopirajte u tekući direktorij. Sadržaj datoteke \texttt{mjesta} sortirajte po abecedi i izlaz upišite u datoteku \texttt{mjesta\_sort} 
\end{zadatak}

\subsection*{Prebacivanje povijesti naredbi na sustav Moodle}

Naredbom \texttt{sharehistory} uploadajte povijest naredbi na web (\texttt{sharehistory} je skripta i nije standardna Linux naredba). Otvorite dobivenu poveznicu u pregledniku i kopirajte povijest u tekstualnu datoteku koju ćete predati na Moodle prema uobičajenim pravilima.
