

\section{Instaliranje Linux operativnog sustava na virtualnu mašinu}
Virtualna mašina (VM) je program koji simulira računalo i na koju možemo instalirati drugi operativni sustav. Operativni sustav (OS) računala zovemo \textit{host} OS, a operativni sustav instaliran na VM zovemo \textit{guest} OS.

U ovoj vježbi instalirat ćemo Linux OS Mint distribuciju koja je bazirana na Ubuntu distribuciji koja je pak bazirana na Debian distribuciji. Sve \textit{Debian-based} distribucije imaju isti sustav pakiranja i instaliranja programa.

\begin{zadatak} Slijediti sljedeće korake instalacije:
\begin{itemize}
\item Odabrati Linux distribuciju, te preuzeti njenu \texttt{.iso} datoteku.
\item Kopirati \texttt{.iso} datoteku na disk.
\item  Pokrenuti Oracle VirtualBox manager.
\item Iz izbornika odabrati Machine, New.
\item Upisati karakteristike virtualnog stroja: Naziv, Tip (Linux), Verzija (Ubuntu 32-bit), RAM (1500 MB)
\item Napraviti virtualni hard disk (VDI, Dynamic, 16 GB)
\item Odabrati ISO sliku: VM settings, Storage, označiti DVD (Empty), Choose Virtual Optical Disk File
%\item Pokrenuti VMWare player ili workspace.
%\item Odabrati opciju: instaliranje novog operativnog sustava.
\item Instalirati operativni sustav. Odabrati \textbf{manualno} određivanje veličine particija.
    
\item Prijedlog veličina particije 

\begin{tabular}{lll}
 \texttt{/} & ext4 & 12GB \\
 \texttt{swap} & linux swap & max 1 GB\\
 \texttt{/home} & ext4 & 3GB 
 \end{tabular}
\item username: student\\
      password: student123  
\end{itemize}
\end{zadatak}
 
 \begin{zadatak} Isprobati funkcionalnosti ditribucije Linux Mint

Postavite prozor virtualnog stroja preko cijelog ekrana \textit{(Full screen mode)}. Sve vježbe ubuduće radit ćemo u Linuxu namještenom preko cijelog ekrana. Pokrenite preglednik i pristupite sustavu Moodle. Isprobajte programe Calculator, Files (\textit{Nemo File Manager}), Double Commander, Gedit (\textit{Text Editor}). Na kraju pokrenite terminal i isprobajte naredbe \texttt{echo} i \texttt{history}.
\\
\\
\texttt{echo "Hello world"}\break
\texttt{echo "Ime Prezime"}\break
\texttt{history}
\\
\\
Usmjerite izlaz naredbe \texttt{history} u datoteku koju ćete predati preko sustava Moodle \textbf{(pažljivo pročitajte upute za vježbe)}:\\
\texttt{history  \textgreater \  naziv\_datoteke}
\\
\\
Pokrenite \textit{file manager}, potražite stvorenu datoteku i pregledajte je u editoru teksta.
Predajte datoteku prema uputama za vježbe. 

\end{zadatak}