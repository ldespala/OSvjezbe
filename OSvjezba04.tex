\section{Wildcards}

Prilikom rada u ljusci mogu se koristiti posebni znakovi za zamjenu jednog ili više znakova u nazivu datoteke ili direktorija.

\begin{center}
\begin{tabular}{lll}
\toprule
znak & značenje & primjer\\


\midrule
\texttt{*} & zamjenjuju grupu znakova bilo koje veličine& \texttt{echo D*}\\
\texttt{?} & zamjenjuje jedan znak&\texttt{echo ?ocument}\\
\texttt{[]} & lista mogućih znakova&\texttt{ls *[tT]}\\ 
\texttt{-} & raspon znakova &\texttt{ls *[1-5]}\\
\texttt{!} &invertira listu znakova&\texttt{ls *[!1-5]}\\
\bottomrule
\end{tabular}
\end{center}
\begin{zadatak} Ispišite sve datoteke u direktoriju \texttt{/usr/bin} koje:
\begin{itemize}
 \item završavaju sa \texttt{config}.
 \item počinju sa \texttt{A} ili \texttt{a}.
 \item završavaju sa znamenkom.
 \item imaju znamenku \texttt{1} na drugoj poziciji.
 \item nemaju znamenku na zadnjoj poziciji.
 \end{itemize}
\end{zadatak}

 \section{Naredbe za pretraživanje datotečnog sustava}
\subsection*{Pretraživanje naredbom \texttt{find}}
Sintaksa: \texttt{find <path> -name <searchstring>}.
\begin{center}
 
\begin{tabularx}{\textwidth}{lXl}
\toprule
opcija & opis & primjer\\
\midrule
\texttt{-name} &pretraživanje po nazivu&\lstinline!find /usr/bin/ -name 'v*'!\\ \addlinespace
\texttt{-iname} &pretraživanje po nazivu (\textit{case insensitive}) &\lstinline!find /usr/bin/ -iname 'v*'!\\ \addlinespace
\texttt{-type} &pretraživanje po tipu&\lstinline!find / -type d -iname '*s'!\\  \addlinespace
\texttt{-mtime} &pretraživanje datoteka modificiranih u zadanom vremenu&\lstinline!find /var/backups -mtime +5!\\ \addlinespace
\texttt{-size} &pretraživanje po veličini u blokovima&\lstinline!find /var/backups -size +50!\\ \addlinespace
\texttt{-newer} &pretraživanje datoteka novijih od navedene&\lstinline!find /var/backups -newer /var/lock!\\ 
\bottomrule
\end{tabularx}

\end{center}

\subsection*{Pretraživanje naredbom \texttt{locate}}

Naredba \texttt{locate} omogućava brzo pronalaženje indeksiranih datoteka u datotečnom sustavu. Primjer: \lstinline!locate password!. Indeksiranje se vrši naredbom \lstinline!updatedb!.

\begin{zadatak} Riješite sljedeće zadatke koristeći \texttt{find} ili \texttt{locate} naredbu:
\begin{enumerate}
 \item Nađite sve datoteke u \texttt{/usr/games} direktoriju čiji nazivi počinju malim slovom.
 \item Nađite sve datoteke u \texttt{home} direktoriju kreirane danas.
 \item Nađite sve datoteke u \texttt{home} direktoriju starije od tjedan dana.
 \item Nađite lokaciju programa \texttt{rmdir}.
 \item Nađite lokaciju datoteke \texttt{.bash\_history}. Što je u njoj?

 \item Pronađite datoteku \texttt{.bash\_logout} naredbom \texttt{locate}.
\item Pronađite sve datoteke u \texttt{home} direktoriju veće od 5MB. Naredba \texttt{find . -size +5000k}.
\item Pronađite sve datoteke u datotečnom sustavu koje u nazivu sadrže riječ \texttt{kernel}. Naredba \texttt{find / -iname *kernel*}

\end{enumerate}
\end{zadatak}
\begin{comment}

\end{comment}




